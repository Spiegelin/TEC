% Options for packages loaded elsewhere
\PassOptionsToPackage{unicode}{hyperref}
\PassOptionsToPackage{hyphens}{url}
%
\documentclass[
]{article}
\usepackage{amsmath,amssymb}
\usepackage{lmodern}
\usepackage{iftex}
\ifPDFTeX
  \usepackage[T1]{fontenc}
  \usepackage[utf8]{inputenc}
  \usepackage{textcomp} % provide euro and other symbols
\else % if luatex or xetex
  \usepackage{unicode-math}
  \defaultfontfeatures{Scale=MatchLowercase}
  \defaultfontfeatures[\rmfamily]{Ligatures=TeX,Scale=1}
\fi
% Use upquote if available, for straight quotes in verbatim environments
\IfFileExists{upquote.sty}{\usepackage{upquote}}{}
\IfFileExists{microtype.sty}{% use microtype if available
  \usepackage[]{microtype}
  \UseMicrotypeSet[protrusion]{basicmath} % disable protrusion for tt fonts
}{}
\makeatletter
\@ifundefined{KOMAClassName}{% if non-KOMA class
  \IfFileExists{parskip.sty}{%
    \usepackage{parskip}
  }{% else
    \setlength{\parindent}{0pt}
    \setlength{\parskip}{6pt plus 2pt minus 1pt}}
}{% if KOMA class
  \KOMAoptions{parskip=half}}
\makeatother
\usepackage{xcolor}
\usepackage[margin=1in]{geometry}
\usepackage{longtable,booktabs,array}
\usepackage{calc} % for calculating minipage widths
% Correct order of tables after \paragraph or \subparagraph
\usepackage{etoolbox}
\makeatletter
\patchcmd\longtable{\par}{\if@noskipsec\mbox{}\fi\par}{}{}
\makeatother
% Allow footnotes in longtable head/foot
\IfFileExists{footnotehyper.sty}{\usepackage{footnotehyper}}{\usepackage{footnote}}
\makesavenoteenv{longtable}
\usepackage{graphicx}
\makeatletter
\def\maxwidth{\ifdim\Gin@nat@width>\linewidth\linewidth\else\Gin@nat@width\fi}
\def\maxheight{\ifdim\Gin@nat@height>\textheight\textheight\else\Gin@nat@height\fi}
\makeatother
% Scale images if necessary, so that they will not overflow the page
% margins by default, and it is still possible to overwrite the defaults
% using explicit options in \includegraphics[width, height, ...]{}
\setkeys{Gin}{width=\maxwidth,height=\maxheight,keepaspectratio}
% Set default figure placement to htbp
\makeatletter
\def\fps@figure{htbp}
\makeatother
\usepackage[normalem]{ulem}
\setlength{\emergencystretch}{3em} % prevent overfull lines
\providecommand{\tightlist}{%
  \setlength{\itemsep}{0pt}\setlength{\parskip}{0pt}}
\setcounter{secnumdepth}{-\maxdimen} % remove section numbering
\ifLuaTeX
  \usepackage{selnolig}  % disable illegal ligatures
\fi
\IfFileExists{bookmark.sty}{\usepackage{bookmark}}{\usepackage{hyperref}}
\IfFileExists{xurl.sty}{\usepackage{xurl}}{} % add URL line breaks if available
\urlstyle{same} % disable monospaced font for URLs
\hypersetup{
  pdftitle={Analogía computadora y la célula},
  pdfauthor={Diego Espejo \& Daniel Esparza},
  hidelinks,
  pdfcreator={LaTeX via pandoc}}

\title{Analogía computadora y la célula}
\author{Diego Espejo \& Daniel Esparza}
\date{2023-04-11}

\begin{document}
\maketitle

\hypertarget{funciones-de-organelos}{%
\subsection{Funciones de organelos:}\label{funciones-de-organelos}}

\begin{description}
\item[\uline{Núcleo}]
El núcleo es una estructura redondeada presente en el centro de la
célula eucariota que contiene el material genético de la célula en forma
de cromosomas. La principal función del núcleo es controlar y regular la
expresión de los genes de la célula.
\item[\uline{Membrana nuclear}]
La membrana nuclear es una estructura de doble capa lipídica que rodea y
protege el núcleo de la célula eucariota. La membrana nuclear controla
el intercambio de sustancias entre el núcleo y el citoplasma de la
célula.
\item[\uline{Nucleolo}]
El nucleolo es una estructura pequeña y redondeada presente en el núcleo
de la célula eucariota que se encarga de la síntesis y ensamblaje de los
componentes ribosomales.
\item[\uline{Ribosomas}]
Los ribosomas son estructuras celulares encargadas de la síntesis de
proteínas a partir de la información genética presente en el ARN
mensajero (ARNm). Los ribosomas pueden encontrarse libres en el
citoplasma o adheridos al retículo endoplásmico.
\item[\uline{Retículo endoplásmico}]
El retículo endoplásmico es un orgánulo formado por una red de membranas
interconectadas que se extiende por todo el citoplasma de la célula. El
retículo endoplásmico se divide en dos tipos: el retículo endoplásmico
rugoso, que contiene ribosomas adheridos a su superficie, y el retículo
endoplásmico liso, que no tiene ribosomas y está especializado en la
síntesis de lípidos y la detoxificación de sustancias nocivas.
\item[\uline{Pared celular/membrana celular}]
La pared celular es una estructura rígida presente en las células de las
plantas, hongos y algunas bacterias que se encarga de proporcionar
protección y sostén a la célula. La membrana celular es una estructura
de doble capa lipídica presente en todas las células que se encarga de
regular el intercambio de sustancias entre el interior y el exterior de
la célula.
\item[\uline{Citoplasma}]
El citoplasma es el espacio que se encuentra entre el núcleo y la
membrana celular de la célula eucariota y que contiene numerosos
orgánulos y estructuras celulares.
\item[\uline{Citoesqueleto}]
El citoesqueleto es un conjunto de filamentos proteicos que se extienden
por todo el citoplasma de la célula eucariota y que se encargan de
proporcionar soporte estructural a la célula, así como de permitir la
movilidad celular y la división celular.
\item[\uline{Mitocondria}]
Las mitocondrias son orgánulos encargados de la producción de energía en
la célula. Las mitocondrias contienen su propio material genético y se
encargan de la síntesis de ATP, la principal fuente de energía de la
célula.
\item[\uline{Aparato de Golgi}]
El aparato de Golgi es un orgánulo formado por una serie de sacos
membranosos apilados que se encargan de modificar, clasificar y empacar
las proteínas y lípidos sintetizados por la célula.
\item[\uline{Membrana nuclear}]
La membrana nuclear es una estructura de doble capa lipídica que rodea y
protege el núcleo de la célula eucariota. La membrana nuclear controla
el intercambio de sustancias entre el núcleo y el citoplasma de la
célula.
\item[\uline{Nucleolo:}]
El nucleolo es una estructura pequeña y redondeada presente en el núcleo
de la célula eucariota que se encarga de la síntesis y ensamblaje de los
componentes ribosomales.
\item[\uline{Ribosomas}]
Los ribosomas son estructuras celulares encargadas de la síntesis de
proteínas a partir de la información genética presente en el ARN
mensajero (ARNm). Los ribosomas pueden encontrarse libres en el
citoplasma o adheridos al retículo endoplásmico.
\item[\uline{Retículo endoplásmico}]
El retículo endoplásmico es un orgánulo formado por una red de membranas
interconectadas que se extiende por todo el citoplasma de la célula. El
retículo endoplásmico se divide en dos tipos: el retículo endoplásmico
rugoso, que contiene ribosomas adheridos a su superficie, y el retículo
endoplásmico liso, que no tiene ribosomas y está especializado en la
síntesis de lípidos y la detoxificación de sustancias nocivas.
\item[\uline{Pared celular/membrana celular}]
La pared celular es una estructura rígida presente en las células de las
plantas, hongos y algunas bacterias que se encarga de proporcionar
protección y sostén a la célula. La membrana celular es una estructura
de doble capa lipídica presente en todas las células que se encarga de
regular el intercambio de sustancias entre el interior y el exterior de
la célula.
\item[\uline{Citoplasma}]
El citoplasma es el espacio que se encuentra entre el núcleo y la
membrana celular de la célula eucariota y que contiene numerosos
orgánulos y estructuras celulares.
\item[\uline{Citoesqueleto}]
El citoesqueleto es un conjunto de filamentos proteicos que se extienden
por todo el citoplasma de la célula eucariota y que se encargan de
proporcionar soporte estructural a la célula, así como de permitir la
movilidad celular y la división celular.
\item[\uline{Mitocondria}]
Las mitocondrias son orgánulos encargados de la producción de energía en
la célula. Las mitocondrias contienen su propio material genético y se
encargan de la síntesis de ATP, la principal fuente de energía de la
célula.
\item[\uline{Aparato de Golgi}]
El aparato de Golgi es un orgánulo formado por una serie de sacos
membranosos apilados que se encargan de modificar, clasificar y empacar
las proteínas y lípidos sintetizados por la célula.
\end{description}

\hypertarget{relaciuxf3n-con-la-computadora}{%
\subsection{Relación con la
computadora}\label{relaciuxf3n-con-la-computadora}}

\begin{description}
\item[\uline{Núcleo - Procesador}]
El procesador de la computadora puede considerarse como el ``núcleo'' de
la misma, ya que es el encargado de controlar y ejecutar las operaciones
lógicas y matemáticas.
\item[\uline{Membrana nuclear - Gabinete}]
La carcasa de la computadora puede asemejarse a la membrana nuclear, ya
que protege los componentes internos de la misma.
\item[\uline{Nucleolo - GPU}]
El microprocesador o CPU de la computadora puede ser comparado con el
nucleolo, ya que se encarga de procesar la información y ejecutar las
instrucciones.
\item[\uline{Ribosomas - Dispositivos de entrada y salida}]
Los dispositivos de entrada y salida de la computadora, como el teclado
y el mouse, pueden ser considerados como los ``ribosomas'' de la misma,
ya que permiten la entrada y salida de información.
\item[\uline{Retículo endoplásmico - Motherboard}]
La motherboard de la computadora puede asemejarse al retículo
endoplásmico, ya que es una estructura que conecta y coordina los
diferentes componentes de la computadora.
\item[\uline{Pared celular - Monitor}]
El monitor de la computadora puede considerarse como la ``pared
celular'' o ``membrana celular'' de la misma, ya que permite visualizar
y controlar la información que se encuentra dentro de la computadora.
\item[\uline{Citoplasma - espacio físico del gabinete}]
El espacio físico dentro del gabinete de la computadora donde se
encuentran los diferentes componentes puede compararse con el
citoplasma, ya que es el espacio que se encuentra entre el núcleo y la
membrana celular.
\item[\uline{Citoesqueleto - Gabinete}]
El gabinete de la computadora puede considerarse como el citoesqueleto,
ya que proporciona soporte estructural y permite la movilidad de los
diferentes componentes.
\item[\uline{Mitocondria - Luz}]
La fuente de alimentación de la computadora puede compararse con las
mitocondrias, ya que se encarga de suministrar energía a la computadora.
\item[\uline{Aparato de Golgi - Disco duro}]
El disco duro o unidad de almacenamiento de la computadora puede
asemejarse al aparato de Golgi, ya que es el encargado de almacenar,
clasificar y organizar la información y los programas.
\end{description}

\begin{longtable}[]{@{}
  >{\raggedright\arraybackslash}p{(\columnwidth - 4\tabcolsep) * \real{0.0872}}
  >{\raggedright\arraybackslash}p{(\columnwidth - 4\tabcolsep) * \real{0.1281}}
  >{\raggedright\arraybackslash}p{(\columnwidth - 4\tabcolsep) * \real{0.7847}}@{}}
\toprule()
\begin{minipage}[b]{\linewidth}\raggedright
Organelo Celular
\end{minipage} & \begin{minipage}[b]{\linewidth}\raggedright
Parte de la Computadora
\end{minipage} & \begin{minipage}[b]{\linewidth}\raggedright
Justificación
\end{minipage} \\
\midrule()
\endhead
Núcleo & Procesador & Ambos son el ``cerebro'' de la célula/computadora,
controlando y ejecutando operaciones lógicas y matemáticas. \\
Membrana Nuclear & Carcasa/Gabinete & La carcasa/gabinete protege los
componentes internos de la computadora, de la misma manera que la
membrana nuclear protege el núcleo y el material genético de la
célula. \\
Nucleolo & CPU/Microprocesador & Ambos procesan información y ejecutan
instrucciones importantes para el funcionamiento de la
célula/computadora. \\
Ribosomas & Dispositivos de entrada/salida & Los ribosomas traducen la
información genética para la síntesis de proteínas, mientras que los
dispositivos de entrada/salida permiten la entrada y salida de
información en la computadora. \\
Retículo Endoplásmico & Placa madre/Motherboard & El retículo
endoplásmico es una red de membranas que conectan y coordinan los
diferentes organelos de la célula, de manera similar a como la placa
madre/motherboard conecta y coordina los diferentes componentes de la
computadora. \\
Pared Celular/Membrana Celular & Monitor & El monitor de la computadora
es la interfaz entre el usuario y la computadora, permitiendo visualizar
y controlar la información en la pantalla, de la misma manera que la
pared celular/membrana celular de la célula controla el flujo de
materiales dentro y fuera de la célula. \\
Citoplasma & Espacio dentro del gabinete de la computadora & El
citoplasma es el espacio dentro de la célula donde se encuentran los
diferentes organelos, de la misma manera que el espacio dentro del
gabinete de la computadora donde se encuentran los diferentes
componentes. \\
Citoesqueleto & Chasis/estructura interna de la computadora & El
citoesqueleto es la red de proteínas que proporciona soporte estructural
y permite la movilidad de los diferentes organelos en la célula, de
manera similar a como el chasis/estructura interna de la computadora
proporciona soporte y permite la movilidad de los diferentes
componentes. \\
Mitocondria & Fuente de alimentación de la computadora & Las
mitocondrias son los organelos encargados de suministrar energía a la
célula, de manera similar a como la fuente de alimentación de la
computadora suministra energía a la misma. \\
Aparato de Golgi & Disco duro/Unidad de almacenamiento & El aparato de
Golgi es el organelo encargado de almacenar, clasificar y organizar la
información y los materiales dentro de la célula, de manera similar a
como el disco duro/unidad de almacenamiento de la computadora almacena,
clasifica y organiza la información y los programas. \\
Lisosomas & Programas antivirus y de limpieza & Los lisosomas son
organelos encargados de la degradación de desechos y materia \\
\bottomrule()
\end{longtable}

\hypertarget{referencias}{%
\subsection{Referencias}\label{referencias}}

Genetics Glossary. (2023). Genome. Recuperado de
\url{https://www.genome.gov/genetics-glossary}

National Human Genome Research Institute. (2023). \emph{Genome.}
Recuperado de
\url{https://www.genome.gov/es/genetics-glossary/Organelo\#:~:text=Un\%20organelo\%20u\%20orgánulo\%20es\%20una\%20estructura\%20específica\%20dentro\%20de,cumplen\%20dentro\%20de\%20una\%20célula.}

\end{document}
